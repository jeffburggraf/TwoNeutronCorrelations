\section{Data Analysis}
\label{Analysis}
\subsection{Calculation of Angular Correlation}
The efficiency and acceptance of the neutron detector array varies greatly over the range of opening angles from 0 to 2$\pi$ (see figure~\ref{fig:Cf252Normalization}(a) ). This effect is due both to the detector array's non-spherical symmetry, and to varying efficiency as a function of particle position.
Knowledge of the array's efficiency as a function of two-neutron opening angle is not needed, because angular correlation is determined by normalizing raw opening angle measurements to an equivalent distribution of uncorrelated neutrons, giving a result that is insensitive to detector efficiencies, while remaining sensitive to angular correlations.
The effect of normalization is illustrated in figure~\ref{fig:Cf252Normalization}. 
Figure~\ref{fig:Cf252Normalization}(a) shows the unnormalized two-neutron opening angle distribution from the spontaneous fission of $^{252}$Cf. The structure is reflective of the geometric acceptance and efficiencies of the neutron detector array. Fig~\ref{fig:Cf252Normalization}(b) shows the normalized distribution. Normalization is achieved by dividing the raw opening angle distribution by the opening angle  distribution formed by uncorrelated two-neutron events in which each neutron is taken from a different fission. When the beam is in use, neutron events from different pulses are used instead. Such neutron pairs will hereafter be referred to as different pulse (DP) pairs.

The neutrons of a DP pair are uncorrelated because events in one pulse do not have casual influence on the events in another pulse.
Detector efficiency and geometry influence same pulse (SP) events and DP events equally.\todo{ToDo: further explain why this statement holds true}
Thus, barring two-neutron correlations and a scaling factor, the DP distribution is identical to the SP distribution.
Each pair of pulses is chosen such that the two pulses occurred within less than a few 100 ms of each other.
This ensures that both pulses are subject to the same experimental conditions, thereby lessening systematic effects from time varying factors such as high-voltage drift, varying beam current, and detector thresholds.
As many pairs as needed can be readily selected until good counting statistics is achieved, because the only restriction for selecting pulse pairs is that they occurred around roughly the same time.
% ToDo: Compare our result for CF252 to another result. Overlay with this figure.
\begin{figure}
    \centering
    \includegraphics[width  = \textwidth ]{Content/Methods/Normalization.png}
    \caption{(a) Unnormalized two-neutron opening angle distribution from the spontaneous fission of $^{252}$Cf (b) The distribution after normalizing to the distribution of uncorrelated two-neutron events.}
    \label{fig:Cf252Normalization}
    \label{fig:Cf252Normalization}
\end{figure}

\subsection{Subtraction of Accidentals}
\label{Subtraction of Accidentals}
An accidental neutron coincidence is defined as a coincidence between two uncorrelated events in a single pulse.
For example, a coincidence between a neutron from a ($\gamma$,1n) reaction and a neutron from photofission.
Another example is a coincidence between two events that are part of the noise background.
In both of these examples, the two events are considered accidentals because they have no causal influence on each another.
A true neutron coincidence, or true for short, is defined here as any pair of correlated neutrons from the same pulse.

Accidentals are removed from the data by subtracting 1/2 times the equivalent distribution formed by the DP data.
The factor of 1/2 arises from the Poissonian statistics that inherently govern all accidentals, whether the accidental events are composed of two neutrons, two photons, two noise events, or any combination thereof\todo{ToDo: Explain why the use of Poissonian statistics is valid. Mention the fact that we only take the first hit}.
An accidental is comprised of the occurrence of two independent events.
Therefore, as per Poissonian statistics, the probability of measuring an accidental in a single pulse is given by:
\begin{displaymath}
SP_{\text{a}} = \frac{e^{-\lambda}\lambda^2}{2} \approx \frac{1}{2}\lambda^{2}
\end{displaymath}
where $SP_{a}$ is the accidental rate of single pulses, $\lambda$ is the mean accidental rate of single pulses–an unknown value.
In this study, the coincidence rates were around $5\times10^{-5}$ events per pulse, so the approximation used above is correct to within 0.001\% as the worst case scenario.
Since the DP data is formed by observations of events from two different pulses, the DP accidental rate is equal to the poissonian probability of observing exactly one event, squared:
\begin{displaymath}
DP_{\text{a}} = (e^{-\lambda}\lambda)^{2}\approx \lambda^{2} 
\end{displaymath}
where $DP_{\text{a}}$ is the accidental rate of DP events.
Therefore, the rate of detected accidentals in single pulses is 1/2 times the rate at which accidentals occur in the different pulse data.
The subtraction of accidentals was about a ten percent effect.
A relatively low accidental rate was the intended outcome of setting the beam current low enough such that there is, on average, less that one fission per pulse.
