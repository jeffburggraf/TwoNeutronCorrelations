\section{Experimental Methods}
\subsection{Photon Beam}
The accelerator's pulse width is set to 3 ns, where as the fastest and slowest neutrons in this experiment have a time of flight of 40 ns and 130 ns, respectively. A bremsstrahlung photon beam is produced by the passage of 10.5 MeV electrons through a 1" thick slab of aluminum. Aluminum was chosen as the radiator because it has a neutron knockout threshold above the energy of the 10.5 MeV electron beam. This prevents the bremsstrahlung radiator from being a source of fast neutrons, which could have the potential to travel into the experimental cell and cause false neutron events.

Downstream from the bremsstrahlung radiator, a sweeping magnet removes excess electrons from the photon beam (see figure~\ref{fig:Facility}). Before reaching the experimental cell, photons are collimated by a series of polyethylene and lead collimators aimed at eliminating beam contaminants.

When attempting a measurement of prompt neutrons from photofission, an ambiguity can arise between neutrons from photofission and neutrons from $(\gamma, xn)$. This is because the two reactions have similar cross-sections within the GDR region. Furthermore, there is significant overlap between the energy spectra of the neutrons from $(\gamma, xn)$ and photofission. Since this measurement is concerned only with observing two neutrons in coincidence, it suffices to set the Bremsstrahlung end-point at 10.5 MeV, since this value is below the threshold for ($\gamma, 2n$) of our targets ($\sim$12 MeV). Using an and-point of 10.5 MeV still leaves the possibility of the detection of multiple ($\gamma, 1n$) reactions in a single pulse, but this is an accidental coincidence. An \textit{accidental} neutron coincidence occurs when two uncorrelated neutrons happen to be detected in the same pulse. All accidentals follow Poissonian distribution, allowing for their subtraction from the data. The details of the procedure used to subtract accidentals is explained in section~\ref{Subtraction of Accidentals}.

The energy distribution of photons reaching the target was simulated using MCNP. The creation and collimation of the Bremsstrahlung photons in included in the simulation. The resulting energy distribution is shown in figure~\ref{fig:BremDist}

\begin{figure}[h]
\includegraphics[width=0.9\textwidth]{Content/Methods/MCNPBremDistribution.png}
\caption{Result of an MCNP simulation of the energy distribution of the Bremsstrahlung photons that reach the target. Points are from the simulation, and the line is an exponential fit ($Ae^{-bx}$). The constant of proportionality, $A$, is arbitrary. The value for $b$ is 0.54.}
\label{fig:BremDist}
\end{figure}

The electron pulse width was set to 3 ns and had a 1.1A peak current, with a repetition rate of 240 Hz. The 3 ns pulse width was not a significant source of error in the measurements of neutron time of flight, since neutron events had a median time of flight of about 80 ns. 
The accelerator current is set by requiring there be, on average, fewer than one fission per pulse, thereby reducing the detection of uncorrelated neutrons from multiple fissions occurring in a single pulse. Even so, the detection of uncorrelated neutrons is unavoidable because of statistical fluctuations. To address this, a technique is developed to subtract these events from the data (see section~\ref{Subtraction of Accidentals}).

(Discussion about the LINAC's low duty factor??)

\subsection{Particle time of flight determination}
\label{reconstruction}

%python file: ProductionAnalysis/TOFGraphs
\begin{figure}[htbp]
\begin{center}

\subfloat[$\Delta T$s with no target in place. ]{\includegraphics[width=0.5\textwidth]{Content/Methods/ToF0.png}}
\end{center}

\subfloat[$\Delta T$s with Aluminum target.]{\includegraphics[width=0.5\textwidth]{Content/Methods/ToF1.png}}
\subfloat[Average of $\Delta T_{\text{top}}$ and $\Delta t_{\text{bot}}$  with Aluminum target in place.]{\includegraphics[width = 0.5\textwidth]{Content/Methods/ToF2.png}}
\caption{The $\Delta T $ spectra of PMTs with no target in place (a) shows the background produced by the beam alone. Using Aluminum as a non-neutron producing target (b), a peak appears in each spectra caused by photons scattering from the target. Since these photons travel a known distance of 1.25 m between the target and detector face, the width of these spectra reflect the range in scintillation light propagation times along the 76.2 cm length of the detector. Taking the average of the $\Delta T$s of the top and bottom PMTs produces a sharp peak (c), since the sum of the scintillation light propagation times from both PMTs is always equal to the time required for light to travel the detector's full length. The correct timing offset can be then be found using using the fact that photons have a time of flight of 4 ns. }
\label{fig:ToFDetermination}
\end{figure}

Each detector was equipped with two PMTs fixed on opposite ends of the scintillation cell, with the exception of the detectors located farthest downstream at $\pm30^{\circ}$, which only had a single PMT. The PMTs provide a signal in response to scintillation light with timing resolution of less than 1 ns. However, the main source of uncertainly in the time of a particle hit is variation in the time taken for scintillation light to propagate to the PMTs. No pulse shape discrimination was used in this study. Particle identification, along with the reconstruction of energy and position was achieved solely from the timing of signals from PMTs. The time of each event in a PMT was measured relative to a signal provided by the accelerator at the beginning of each pulse, which is referred to as the \textit{beam gun}.  

Time of flight (ToF), the time for a particle to travel from the target to the face of a detector, was used to distinguish between photons and neutrons, and to measure neutron energy. The time of flight was calculated by taking the average between the times of signals from the top and bottom PMTs, and subtracting an offset determined from a calibration. In the detectors located at $\pm30^{\circ}$, which have only one PMT, ToF is calculated from the timing of events in its sole PMT, minus a calibration offset.

The ToF of a particle that causes coincident events in both PMTs of a detector, obeys the following relationship:
\begin{displaymath}
ToF = C_i + \Delta t_{\text{avg}} 
\end{displaymath}
where $\Delta t_{\text{avg}} $ is the average between the timing from the top and bottom PMTs, and $C_i$ is a constant timing offset that is the same for every pulse. The subscript on $C_i$ is used because the timing offset is different for each detector. Any process that produces a timing delay that does not change from pulse to pulse contributes to $C_{i}$. Examples of this are the time required for photons to travel from the bremsstrahlung radiator to the target, the propagation of signals through the wires connecting the PMTs, and delays in the electronics for processing. 

The time required for scintillation light to travel through the detector, from point of scintillation to a PMT mounted at the detector's top or bottom, can vary from 1 ns for particles that hit very close to a given PMT, to about 8 ns for particles that hit across the detector from a given PMT. The sum of the times taken for scintillation light to travel to the top and bottom PMTs is just the time taken for the light to travel the full length of the detector, which is nearly constant. The rate at which light propagates along the length of a detector is dependant on speed of light in the material and the light's flight path. The flight paths of detected scintillation light tend to be parallel to the long axis of the detector, because these paths are the shortest possible, and only the first signal from a PMT is accepted. Therefore, by taking the average of the times in the top and bottom PMT, the time required for scintillation light to propagate through the scintillator becomes a constant offset.

The validity of a constant scintillation light propagation time was verified experimentally using a $^{60}Co$ source. With the face of the scintillator covered with 2" thick bricks of lead shielding, the $^{60}Co$ source is placed against the lead at several locations along the length of the detector. The lead brick between the $^{60}Co$ source and the scintillator has a small hole drilled through it, giving the $^{60}Co$ a direct line of sight to a point on the face of the scintillator.  
%TODO: Make a figure for the co60 calibration. 

\begin{figure}
    \centering
    \includegraphics[width = 0.9\textwidth]{Content/Methods/CO60Validation.png}
    \caption{A $^{60}Co$ source, which emits coincident back-to-back photons, is placed at several positions along the face of a lead shielded scintillator. At each position, a small hole is drilled through the lead to give the $^{60}Co$ source a line-of-sight to a well-defined point on the scintillator. Then, a high timing resolution photon detector is placed close to the $^{60}Co$ source. $^{60}Co$ decays to emit two photons simultaneously--one photon is detected by the high timing resolution detector and serves as the ``start'' time, and the other photon causes a scintillation event in the detector being calibrated .}
    \label{fig:Co60Validation}
\end{figure}

\begin{figure}
    \centering
    \includegraphics[width = 0.8\textwidth]{Content/Methods/lightpaths.png}
    \caption{Hypothetical paths of light rays as they propagate through the scintillator after a scintillation event. The light rays that first reach a PMT (solid) travel a shorter path than the others (dotted), and thus experience the least amount of attenuation. This experiment always uses the first signal from a PMT and hence the data favors cases in which scintillation light travels a short path to a PMT. In other words, the detectors favor the detection of scintillation light that is traveling parallel to the long axis of the detector. As a result, the sum of the times required for scintillation light to reach both PMTs is just the time taken for light to travel the full length of the detector, which is a constant. Thus, by taking the average time between coincident events in both PMTs, the time required for the propagation of scintillation light becomes a constant that is removed during calibration. }
    \label{fig:lightpath}
\end{figure}

The value of the constant offset for the calculation of ToF is determined by observing photons that are known to have scattered from the target. Comparing the timing spectra of a non neutron producing target made from aluminum, to the spectra produced when no target is used reveals a prominent peak caused by the photons that scatter from the target. The distance these photons must travel to reach the center of a face of any detector is 125 cm, and 130 cm to reach the top or bottom edges of a detector. It takes light 4.1 ns and 4.3 ns to travel 125 cm and 130 cm, respectively. The difference between these two ToFs is negligible compared to experimental uncertainty, so the ToF of photons that scatter from the target is assumed to be 4 ns, and with this assumption the constant timing offset of each detector can be calculated.  
   
\subsection{Particle position reconstruction}
All that is known about a particle's position projected onto the horizontal plane, is that the particle is within scintillation cell of the detector in which it was detected. The detector's dimensions in the horizontal plane are comparatively small at 5X15 cm$^2$, so it suffices to use the geometric center of the detector as this component of a particle's position. Doing this creates a positional uncertainty of $\pm$8 cm, which in terms of an opening angle, amounts to $\pm4^{\circ}$. The final results of this work use an opening angle bin width of 20$^{\circ}$, so $\pm4^{\circ}$ isn't a large component of uncertainty. The largest contributor to uncertainty in the reconstruction of particle position is the position in vertical direction, which is determined by the timing difference between signals in the top and bottom PMTs. 

As is the case with ToF determination, particle position in the vertical direction relies on the timing of coincident signals from both the PMTs of a detector. If the signals from each PMT are the result of the detection of scintillation light created by the same particle, then the timing difference between the signals obeys a linear relationship with respect to the position of the particle along the vertical axis of the detector, hereafter referred to as the z-axis where the geometrical center of the detector is at $z=0$. As discussed above, scintillation light that is detected tends to travel a relatively straight path to the PMT, with minimal reflections off the boundary of the scintillation cell. Because of this, the timing difference between signals in the top and bottom PMTs is proportional to the difference in path the lengths that the scintillation light must travel to reach each PMT. The difference in paths lengths is proportional to the z coordinate, thus the timing difference is also proportional to the z coordinate. The exact linear relationship is determined through calibration by using collimated photons from a $^{60}$Co. The calibration consists of measuring the top-bottom timing difference with the source of collimated photons fixed at five different locations along the detectors length. The result is shown in figure~\ref{fig:PMTDifference}. The same setup used here was also used for ToF calibration as discussed in section~\ref{reconstruction}, in which a two inch thick brick of lead that has a hole drilled through it, about the diameter of a pencil, in order to produce a point beam of photons incident on the detector at a precise location. Noise is completely eliminated by the requirement that there be 3-fold coincidence between signals from the top and bottom PMTs of the detector being calibrated and a signal from the trigger detector. The rate. 
\begin{figure}
    \centering
    \includegraphics[width = 0.9\textwidth]{Content/Methods/PMTDifference.png}
    \caption{Caption}
    \label{fig:PMTDifference}
\end{figure}
\subsection{Detector Cross-talk}
Shielding.
POLIMI Simulation.

\subsection{Targets}
A depleted Uranium (DU) target with dimensions of XbyYbyZ $cm^3$ was used as the primary target for the measurement of two-neutron correlations. DU received the majority of the allotted beam time because it is an even-even nuclei, and as a consequence, always has exactly one quanta of angular momentum immediately upon the absorption of a photon. This makes it a good nuclei for photofission as a means to study the fundamentals of fission. 

One consideration for the dimensions of the target is the rate at which fission neutrons produced in the target scatter before exiting the target. This has the potential to be a major problem, since a scattering event changes the neutron's direction of the travel and thus creates two-neutron opening angles that are not reflective of what the opening angle was immediately after fission. This effect cannot be eliminated completely, but the target must have small enough dimensions such that the effects of neutron scattering can be neglected. The probability that a two-neutron event is undisturbed by scattering is calculated by squaring the probability that a single neutron is not disturbed. The probability that a single neutron is not disturbed is calculated by an MCNP simulation in which neutrons with an energy spectrum typical of fission neutrons are sample uniformly within a target.

It is desirable to have a target with a symmetry that is congruent with the symmetry of the cylindrically structured neutron detector array. There were difficulties in obtaining a cylindrical target, so instead a thin rectangular was rotated slowly during data acquisition. In doing so, cylindrical symmetry is preserved in the final result, since the final result is an average taken from events which occurred over a long period of time. 
The final target design was tested for neutron scattering with an MCNP simulation. In the simulation, the target cylindrical source of photons flooded inward at the target. The probability that a photo-neutron produced in the target escaped without scattering was 97.5\%. Because two neutrons are required for the formation of an opening angle, the rate of data contamination due to scattering is $(1-.975^2)$, or 5\% of two-neutron events.  

