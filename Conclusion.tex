\chapter{Concluding Remarks}
\thispagestyle{fancy}
Neutron-neutron angular correlations in the photofission of $^{238}$U were measured using 10.5 MeV end-point bremsstrahlung photons produced via a low duty factor, pulsed linear electron accelerator.
The measured angular correlations reflect the underlying back-to-back nature of the fission fragments.
The method of analysis used a single set of experimental data to produce a opening angle distribution of correlated and uncorrelated neutron pairs.
A ratio is taken between these two sets to provide a self-contained result of angular correlations, in that the result is independent of neutron detector efficiencies.
Neutron-neutron angular correlation measurements were also made using neutrons from the spontaneous fission of $^{252}$Cf and show good agreement with previous measurements.

Measured n-n opening angle correlations in the photofission of $^{238}$U do not agree very well with simulations using FREYA version 2.0.3, which uses a neutron-induced model to approximate photofission.
These data will be useful for fine-tuning the photofission models that will be incorporated into future versions of FREYA.

In addition, we report for the first time a pronounced anomaly in the n-n angular distributions from photofission, in which the rate of neutron emission at opening angles near 180$^{\circ}$ is diminished, resulting in a local maximum at about 160$^{\circ}$ instead of the expected 180$^{\circ}$.
We offer one possible explanation for this effect. %relating to the unique feature of the asymmetric angular emission of fission fragments in photofission.
The neutrons may indeed be emitted isotropically in the rest frame of each fission fragment, but there is a decrease in neutron emission along the fission axis because one fragment essentially shadows the neutrons emitted from the other fragment, either through absorption or scattering.
While these measurements do not provide a definitive interpretation of this decreased n-n correlation for large opening angles in photofission, further study has the potential to shed light on the time evolution of neutron emission in photofission.

These first measurements of n-n correlations in photofission will provide the impetus for future modeling of the fundamental physics of fission.