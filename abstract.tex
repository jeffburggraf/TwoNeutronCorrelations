\begin{abstract}
Past study of fission reactions has established that fission neutrons are predominantly emitted by the fully accelerated
fission fragments, as opposed to being emitted during scission. The velocities of the two fully accelerated fission fragments is of similar magnitude to the velocities of the fission neutrons as they are emitted in the fragment’s rest frame. Thus, the motion of the fragments
has a large effect on the kinematics of fission neutrons. This can be seen in the opening angle
distributions of correlated neutron pairs from individual fission events. This effect has been measured multiple times for the spontaneous fission of $^{252}$Cf and the thermal induced fission of $^{235}$U with relatively good agreement. A primary motivation for this work is that to date there have been no reported measurements of this type with photofission. A project has been completed at the Idaho Accelerator Center to measure two-neutron correlations in photofission using bremsstrahlung photons produced via a low duty factor linear electron accelerator. The bremsstrahlung photons impinge upon an actinide target that is surrounded by a large neutron scintillation detector array capable of measuring particle position and time of flight, enabling the calculation of two-neutron opening angle and neutron energy. Correlated distributions in two-neutron opening angle, the angle between a neutron and the incident photon beam, and neutron energy are extracted from the data.
\end{abstract}
