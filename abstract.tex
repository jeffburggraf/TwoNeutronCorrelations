\begin{abstract}
 In the fission of actinides, the nearly back-to-back motion of the fission fragments has a strong effect on the kinematics of fission neutrons.
This effect is seen in the two-neutron opening angle distributions of correlated neutron pairs from the same fission event in which a favoring of opening angles near 0$^{\circ}$ and 180$^{\circ}$ is observed.
       The purpose of this work is to measure the two-neutron opening angle and energy correlations in the photofission of $^{238}$U.
        As of this writing, correlated two-neutron opening angle distributions have been measured using neutrons from spontaneous and neutron-induced fission of actinides.
        This work is the first to report such a measurement using photofission, and will provide useful experimental input for photofission models used in codes such as MCNP and FREYA.

        Fission is induced using bremsstrahlung photons produced via a low duty factor, pulsed, linear electron accelerator.
        The photon beam impinges upon a $^{238}$U target that is surrounded by a large neutron scintillation detection system capable of measuring particle position and time of flight, from which two-neutron opening angle and energy is calculated.
        Two-neutron angular correlations are determined by taking the ratio of the opening angle distribution of correlated neutron pairs to that of uncorrelated neutrons pairs in which the neutrons of each pair were detected during different accelerator pulses.
        This technique gives a self-contained result which greatly diminishes effects due to detector efficiencies, acceptance, and experimental drifts.

       The angular correlation of neutrons from the photofission of $^{238}$U shows a high dependence on neutron energy and a slight dependence on the angle of the emitted neutrons with respect to the incoming photon beam.
        Angular correlations were also measured using neutrons from the spontaneous fission of $^{252}$Cf, showing good agreement with past measurements.
        An anomalous decline in two-neutron yield was observed for opening angles near 180$^{\circ}$.
\end{abstract}
