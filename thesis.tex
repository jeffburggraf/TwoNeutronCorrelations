%% (Master) Thesis template
% Template version used: v1.4
%
% Largely adapted from Adrian Nievergelt's template for the ADPS
% (lecture notes) project.



%% We use the memoir class because it offers a many easy to use features.
\documentclass[11pt,a4paper,titlepage]{memoir}

%% Packages
%% ========

% for revision
\usepackage{float}
%% LaTeX Font encoding -- DO NOT CHANGE
\usepackage[OT1]{fontenc}

%% Babel provides support for languages.  'english' uses British
%% English hyphenation and text snippets like "Figure" and
%% "Theorem". Use the option 'ngerman' if your document is in German.
%% Use 'american' for American English.  Note that if you change this,
%% the next LaTeX run may show spurious errors.  Simply run it again.
%% If they persist, remove the .aux file and try again.
\usepackage[american]{babel}

%% Input encoding 'utf8'. In some cases you might need 'utf8x' for
%% extra symbols. Not all editors, especially on Windows, are UTF-8
%% capable, so you may want to use 'latin1' instead.
\usepackage[utf8]{inputenc}

%% This changes default fonts for both text and math mode to use Herman Zapfs
%% excellent Palatino font.  Do not change this.
\usepackage[sc]{mathpazo}

%% The AMS-LaTeX extensions for mathematical typesetting.  Do not
%% remove.
\usepackage{amsmath,amssymb,amsfonts,mathrsfs}

%% NTheorem is a reimplementation of the AMS Theorem package. This
%% will allow us to typeset theorems like examples, proofs and
%% similar.  Do not remove.
%% NOTE: Must be loaded AFTER amsmath, or the \qed placement will
%% break
\usepackage[amsmath,thmmarks]{ntheorem}

%% LaTeX' own graphics handling
\usepackage{graphicx}

%% We unfortunately need this for the Rules chapter.  Remove it
%% afterwards; or at least NEVER use its underlining features.
\usepackage{soul}

%% This allows you to add .pdf files. It is used to add the
%% declaration of originality.
\usepackage{pdfpages}

%% Some more packages that you may want to use.  Have a look at the
%% file, and consult the package docs for each.
\input{extrapackages}

%% Our layout configuration.  DO NOT CHANGE.
\input{layoutsetup}

%% Theorem environments.  You will have to adapt this for a German
%% thesis.
\input{theoremsetup}

%% Helpful macros.
\input{macrosetup}

%% Make document internal hyperlinks wherever possible. (TOC, references)
%% This MUST be loaded after varioref, which is loaded in 'extrapackages'
%% above.  We just load it last to be safe.
\usepackage[linkcolor=black,colorlinks=true,citecolor=black,filecolor=black]{hyperref}

\usepackage[export]{adjustbox}


\usepackage{subfig}
\usepackage{placeins}
\usepackage{scalerel}



%% Document information
%% ====================

\title{Two-Neutron Correlations in the Photofission of Actinides}
\author{Jeffrey S. Burggraf}
\thesistype{phd Thesis}
\advisors{Advisors: Prof.\ Dr.\ D. S. Dale}
\department{Department of Physics}
\date{January 19, 2018}

% double spacing for comments. 
\DisemulatePackage{setspace}
\usepackage{setspace} 
\setstretch{1.75}
\usepackage[margin=1.5in]{geometry}
\usepackage[inline]{enumitem}
\usepackage{xcolor}
\usepackage[textsize=small]{todonotes}
%
\setlength{\marginparwidth}{2.5cm}

\begin{document}

\frontmatter

%% Title page is autogenerated from document information above.  DO
%% NOT CHANGE.
\begin{titlingpage}
  \calccentering{\unitlength}
  \begin{adjustwidth*}{\unitlength-24pt}{-\unitlength-24pt}
    \maketitle
  \end{adjustwidth*}
\end{titlingpage}

%% The abstract of your thesis.  Edit the file as needed.
\begin{abstract}
\addcontentsline{toc}{chapter}{Abstract}

 In the fission of actinides, the nearly back-to-back motion of the fission fragments has a strong effect on the kinematics of fission neutrons.
This effect is seen in the neutron-neutron opening angle distributions of correlated neutron pairs from the same fission event in which a favoring of opening angles near 0$^{\circ}$ and 180$^{\circ}$ is observed.
As of this writing, correlated neutron-neutron opening angle distributions have been measured using neutrons from spontaneous and neutron-induced fission of actinides.
This work is the first to report such a measurement using photofission, and will provide useful experimental input for photofission models used in codes such as MCNP and FREYA.

   Fission is induced using bremsstrahlung photons produced via a low duty factor, pulsed, linear electron accelerator.
        The bremsstrahlung photon beam impinges upon a $^{238}$U target that is surrounded by a large neutron scintillation detection system capable of measuring particle position and time of flight, from which n-n opening angle and energy are measured.
Neutron-neutron angular correlations are determined by taking the ratio between a correlated neutron distribution and an uncorrelated neutron distribution formed by the pairing of neutrons produced during different beam pulses.
        This analysis technique greatly diminishes effects due to detector efficiencies, acceptance, and experimental drifts.

      The angular correlation of neutrons from the photofission of $^{238}$U shows a high dependence on neutron energy as well as a dependence on the angle of the emitted neutrons with respect to the incoming photon beam.
        Angular correlations were also measured using neutrons from the spontaneous fission of $^{252}$Cf, showing good agreement with past measurements.
        An anomalous decline in neutron-neutron yield was observed for opening angles near 180$^{\circ}$.
\end{abstract}


%% TOC with the proper setup, do not change.
\cleartorecto
\tableofcontents
\mainmatter


\chapter{Introduction}
\thispagestyle{empty}

%\section{Introduction}
%\label{sec:level1}
%\input{../nnCorrPhysRev/Intro.tex}
\chapter{Methods}
\thispagestyle{fancy}


\section{Apparatus}
This experiment was carried out at the Idaho Accelerator Center (IAC), using their fast-pulsed linear accelerator, which is an L--band frequency (1300 MHz) electron linear accelerator.
It is capable of pulse widths ranging from 50 ps to 2 $\mu$s with a maximum energy of 44 MeV.
See section~\ref{beam} for the accelerator parameters used during the experiment.
Figure~\ref{fig:Facility} shows a top down diagram of the experimental arrangement.

\begin{figure}[h]
\centering
\includegraphics[width=0.95\textwidth]{Content/Methods/ExpArangment.jpg}
\caption{To-scale, top down diagram of the experimental setup.
An electron beam impinges upon a 3.8 cm thick Al radiator, and the resulting bremsstrahlung beam enters the experimental cell from the top.
The supporting structure for each detector has been labeled according to the angle, in degrees, between the center of each detector and direction of the incoming photon beam.
}
\label{fig:Facility}
\end{figure}
\subsection{Detectors}
\label{subsection:detectors}
The detection system measures neutron position and time of flight (ToF), which is defined as the time taken for a particle to travel from the target to any detector.
The purpose of the ToF measurement is to determine the kinetic energy of detected neutrons and to distinguish between photons and neutrons.
The neutron detection system consists of fourteen shielded scintillators arranged in a ring around the target (see Fig.~\ref{fig:DetGeom}).
The scintillators were made from Polyvinyl Toluene (PVT), an organic plastic scintillator.
Attached to both ends of each scintillator are 10-cm long, non-scintillating, ultra-violet transmitting plastic light guides.
A Hamamatsu 580-17 photomultiplier (PMT) tube is fixed to each light guide using optical glue.
In order to increase the chance that scintillation light remains inside the scintillator, they were polished to remove micro-imperfections and were wrapped in reflective aluminized mylar.

\begin{figure}[]
    \centering
    \includegraphics[width = 0.9\textwidth]{Content/Methods/Detectors.png}
    \caption{3-D render of the bare, unshielded scintillators, along with PMTs and light guides.}
    \label{fig:DetGeom}
\end{figure}

Ten out of the fourteen scintillators had dimensions of 76.2$\times$15.2$\times$3.8 cm$^3$.
The remaining four scintillators, with dimensions of 25.4$\times$15.2$\times$3.8 cm$^3$, were located at 30$^{\circ}$ and 330$^{\circ}$ with respect to the beam.
These scintillators were segmented in order to address very high photon detection rates resulting from the forward scattering of photons from the target.
Prior to segmentation, a photon was registered in these detectors nearly every pulse, and because the electronics were operated in single hit mode, the detection of a photon leaves the detector unable to detect subsequent neutrons, reducing the neutron detection efficiency to nearly 0\%.
After segmentation, the photon detection rate fell to 0.5 photons per pulse, greatly improving detector live-time.
The detectors at $\pm$30 degrees also differ from the rest in that they were instrumented with only a single PMT, and therefore have a lower position/energy resolution than the others.
In order to test for systematic errors that may have resulted from the use of the segmented detectors, opening angle measurements were compared with and without their use, and the differences were well within experimental errors.

The relative efficiencies of the neutron detectors as a function of neutron energy were calculated by dividing measured and theoretical yields from the SF of $^{252}$Cf.
The results are shown in Fig.~\ref{fig:RelErgEfficiency}, which uses the aggregate of events from all detectors, and in Fig.~\ref{fig:RelErgEfficiencyVariation}, which shows it for each detector individually.
See section~\ref{Analysis} for a discussion on how the effects of detector efficiency are accounted for in this work.
\begin{figure}[]
    \centering
    \includegraphics[width = 0.9\textwidth]{Content/Methods/RelErgEfficiency.png}
    \caption{The relative efficiency of the neutron detection system as a function of neutron energy is calculated by dividing the measured energy distribution by the theoretical energy distribution of neutrons from the SF of $^{252}$Cf.}
    \label{fig:RelErgEfficiency}
\end{figure}
\begin{figure}[]
    \centering
    \includegraphics[width = 1\textwidth]{Content/Methods/RelErgEfficiencyVariation.png}
    \caption{
    The neutron detection efficiency as a function of neutron energy varies among the detectors, which are labeled in this figure according to the angle each detector from the direction of the bremsstrahlung photon beam.
    The y-axis has arbitrary units, and thus each curve is scaled to have the same integral.
    For the integrated neutron rates of each detector, see table~\ref{table:rates} in the appendix.}
    \label{fig:RelErgEfficiencyVariation}
\end{figure}

\subsection{Detector Shielding}
\label{shielding}
\begin{figure}
    \centering
    \includegraphics[width = 0.65\textwidth]{Content/Methods/DetShielding.png}
    \caption{Detector shielding was designed to reduce the detection of photons, room return, and detector cross-talk.}
    \label{fig:shielding}
\end{figure}
The detector shielding, depicted in Fig.~\ref{fig:shielding}, was constructed using lead and polyethylene with the aim of reducing cross-talk, the detection of photons, and noise.
Pb was used to attenuate photons, but has the side effects of neutron scattering and reduced neutron detection efficiency.
If a neutron scatters prior to being detected, the opening angle reconstruction and ToF calculation will be incorrect because both assume that detected neutrons travel a straight path from the target to the detector.
2.5~cm of Pb was placed along the front face of the scintillators, which is enough to largely reduce photon detection rates, and, according to an MCNP simulation, leads to a root-mean-square error in opening angle and ToF of 1$^{\circ}$ and 0.3~ns, respectively, due to neutron elastic scattering.
The sides of each scintillator were shielded with 5 cm of Pb to attenuate photons, followed by 5 cm of polyethylene to reduce the chance of neutron cross-talk.
Also in order to minimize cross-talk, Pb was not placed behind the scintillators after an MCNP-POLIMI simulation indicated it would occur at significant rates otherwise.
Instead, 10~cm of polyethylene was placed behind the scintillators.
For a more detailed discussion about the issue of cross-talk, see section~\ref{crosstalk}.

\subsection{Bremsstrahlung Photon Beam}
\label{beam}
In order to ensure that all correlated neutrons produced are due to fission, the bremsstrahlung end-point was set to 10.5~MeV, safely below the ($\gamma, 2n$) threshold of 11.28~MeV for $^{238}$U.
Al was chosen for a bremsstrahlung radiator, because Al has a neutron knockout threshold above the energy of the electron beam, which ensured that the radiator would not be a source of fast neutrons with the potential to interfere with the experiment.
Downstream from the bremsstrahlung radiator is a sweeping magnet to remove charged particles from the photon beam.
Next, the beam traveled through a series of polyethylene and lead collimators and into the experimental cell in which the target was located (see Fig.~\ref{fig:Facility}).
Figure~\ref{fig:BremDist} shows the energy distribution of photons that reach the target according to an MCNP simulation that included the production and collimation of the bremsstrahlung photon beam.

The electron beam pulse width was set to 3~ns with a repetition rate of 240~Hz and a 1.1~A peak current.
The 3~ns pulse width was small compared to the median neutron ToF of 80~ns, and thus made a small contribution to the uncertainty in the neutron energy determination.

\begin{figure}[h]
\centering
\includegraphics[width=0.7\textwidth]{Content/Methods/MCNPBremDistribution.png}
\caption{MCNP simulation of the energy distribution of photons that are incident on the fission target.}
\label{fig:BremDist}
\end{figure}

\subsection{DU Target}
\label{subsection:targets}
\begin{figure}[]
\centering
    \subfloat[]{\includegraphics[width=0.5\textwidth]{Content/Methods/MTvsAl.png}}
    \subfloat[]{\includegraphics[width=0.5\textwidth]{Content/Methods/DUvsAl.png}}
    \caption{(a) Comparison between the ToF spectrum of a non-neutron producing target made from Al, to the ToF spectrum produced when no target is used.
    The large increase in events around 4~ns is due to photons that scatter from the Al target.
    When no target is in place, sources of the peak include the collimator leading into the experimental cell and the beam dump.
    The photon peak seen here is used to find the timing offsets that make it so $t=0$ corresponds to the moment of fission.
    (b) Comparison between the Al and DU targets show a pronounced increase in events between 35 and 130~ns due to the introduction of neutrons.}
    \label{fig:ToF}
\end{figure}
A depleted uranium (DU) target with dimensions of 4$\times$2$\times$0.05 $\text{cm}^3$ was used as the primary target.
DU received the majority of the allotted beam time because it is an even-even nucleus, and as a consequence, the fission fragments are emitted with a high degree of anisotropy~\cite{1977FragAss}.
Because the target lacked cylindrical symmetry, it was rotated slowly about the vertical axis during the experiment.
This was done in order to remove the potential for bias due to the elastic scattering of neutrons within the target.
See section~\ref{subsection:Elastic_scattering} for details.

\subsection{Electronics}
A data acquisition system based on NIM/VME standard was used.
A schematic of the data acquisition logic is shown in Figure~\ref{fig:WiringDiagram}.
The PMTs are supplied negatgive voltages ranging from 1300 to 1500 V by a LeCroy 1458 high voltage mainframe.
Analog signals from the PMTs were fed into a leading edge discriminator with input thresholds ranging from 30 mV to 50 mV.
The logic signals from the discriminator were then converted to ECL logic and fed into a CAEN model V1290A TDC.
The timing of signals from the PMTs were always measured relative to a signal from the accelerator provided at the beginning of each pulse.
Only the first signal from a given PMT from each pulse is accepted.
On the software side, the CODA~\cite{CODA} software package developed by Jefferson Laboratory was used to read out the acquisition of data from the TDC and convert it into a usable format.

\begin{figure}[h]
\includegraphics[width=0.9\textwidth]{Content/Methods/WiringDiagram.png}
\caption{Wiring diagram of the electronics setup. }
\label{fig:WiringDiagram}
\end{figure}

\section{Experimental Methods}
\subsection{Photon Beam}
The accelerator's pulse width is set to 3 ns, where as the fastest and slowest neutrons in this experiment have a time of flight of 40 ns and 130 ns, respectively. A bremsstrahlung photon beam is produced by the passage of 10.5 MeV electrons through a 1" thick slab of aluminum. Aluminum was chosen as the radiator because it has a neutron knockout threshold above the energy of the 10.5 MeV electron beam. This prevents the bremsstrahlung radiator from being a source of fast neutrons, which could have the potential to travel into the experimental cell and cause false neutron events.

Downstream from the bremsstrahlung radiator, a sweeping magnet removes excess electrons from the photon beam (see figure~\ref{fig:Facility}). Before reaching the experimental cell, photons are collimated by a series of polyethylene and lead collimators aimed at eliminating beam contaminants.

When attempting a measurement of prompt neutrons from photofission, an ambiguity can arise between neutrons from photofission and neutrons from $(\gamma, xn)$. This is because the two reactions have similar cross-sections within the GDR region. Furthermore, there is significant overlap between the energy spectra of the neutrons from $(\gamma, xn)$ and photofission. Since this measurement is concerned only with observing two neutrons in coincidence, it suffices to set the Bremsstrahlung end-point at 10.5 MeV, since this value is below the threshold for ($\gamma, 2n$) of our targets ($\sim$12 MeV). Using an and-point of 10.5 MeV still leaves the possibility of the detection of multiple ($\gamma, 1n$) reactions in a single pulse, but this is an accidental coincidence. An \textit{accidental} neutron coincidence occurs when two uncorrelated neutrons happen to be detected in the same pulse. All accidentals follow Poissonian distribution, allowing for their subtraction from the data. The details of the procedure used to subtract accidentals is explained in section~\ref{Subtraction of Accidentals}.

The energy distribution of photons reaching the target was simulated using MCNP. The creation and collimation of the Bremsstrahlung photons in included in the simulation. The resulting energy distribution is shown in figure~\ref{fig:BremDist}

\begin{figure}[h]
\includegraphics[width=0.9\textwidth]{Content/Methods/MCNPBremDistribution.png}
\caption{Result of an MCNP simulation of the energy distribution of the Bremsstrahlung photons that reach the target. Points are from the simulation, and the line is an exponential fit ($Ae^{-bx}$). The constant of proportionality, $A$, is arbitrary. The value for $b$ is 0.54.}
\label{fig:BremDist}
\end{figure}

The electron pulse width was set to 3 ns and had a 1.1A peak current, with a repetition rate of 240 Hz. The 3 ns pulse width was not a significant source of error in the measurements of neutron time of flight, since neutron events had a median time of flight of about 80 ns. 
The accelerator current is set by requiring there be, on average, fewer than one fission per pulse, thereby reducing the detection of uncorrelated neutrons from multiple fissions occurring in a single pulse. Even so, the detection of uncorrelated neutrons is unavoidable because of statistical fluctuations. To address this, a technique is developed to subtract these events from the data (see section~\ref{Subtraction of Accidentals}).

(Discussion about the LINAC's low duty factor??)

\subsection{Particle time of flight determination}
\label{reconstruction}

%python file: ProductionAnalysis/TOFGraphs
\begin{figure}[htbp]
\begin{center}

\subfloat[$\Delta T$s with no target in place. ]{\includegraphics[width=0.5\textwidth]{Content/Methods/ToF0.png}}
\end{center}

\subfloat[$\Delta T$s with Aluminum target.]{\includegraphics[width=0.5\textwidth]{Content/Methods/ToF1.png}}
\subfloat[Average of $\Delta T_{\text{top}}$ and $\Delta t_{\text{bot}}$  with Aluminum target in place.]{\includegraphics[width = 0.5\textwidth]{Content/Methods/ToF2.png}}
\caption{The $\Delta T $ spectra of PMTs with no target in place (a) shows the background produced by the beam alone. Using Aluminum as a non-neutron producing target (b), a peak appears in each spectra caused by photons scattering from the target. Since these photons travel a known distance of 1.25 m between the target and detector face, the width of these spectra reflect the range in scintillation light propagation times along the 76.2 cm length of the detector. Taking the average of the $\Delta T$s of the top and bottom PMTs produces a sharp peak (c), since the sum of the scintillation light propagation times from both PMTs is always equal to the time required for light to travel the detector's full length. The correct timing offset can be then be found using using the fact that photons have a time of flight of 4 ns. }
\label{fig:ToFDetermination}
\end{figure}

Each detector was equipped with two PMTs fixed on opposite ends of the scintillation cell, with the exception of the detectors located farthest downstream at $\pm30^{\circ}$, which only had a single PMT. The PMTs provide a signal in response to scintillation light with timing resolution of less than 1 ns. However, the main source of uncertainly in the time of a particle hit is variation in the time taken for scintillation light to propagate to the PMTs. No pulse shape discrimination was used in this study. Particle identification, along with the reconstruction of energy and position was achieved solely from the timing of signals from PMTs. The time of each event in a PMT was measured relative to a signal provided by the accelerator at the beginning of each pulse, which is referred to as the \textit{beam gun}.  

Time of flight (ToF), the time for a particle to travel from the target to the face of a detector, was used to distinguish between photons and neutrons, and to measure neutron energy. The time of flight was calculated by taking the average between the times of signals from the top and bottom PMTs, and subtracting an offset determined from a calibration. In the detectors located at $\pm30^{\circ}$, which have only one PMT, ToF is calculated from the timing of events in its sole PMT, minus a calibration offset.

The ToF of a particle that causes coincident events in both PMTs of a detector, obeys the following relationship:
\begin{displaymath}
ToF = C_i + \Delta t_{\text{avg}} 
\end{displaymath}
where $\Delta t_{\text{avg}} $ is the average between the timing from the top and bottom PMTs, and $C_i$ is a constant timing offset that is the same for every pulse. The subscript on $C_i$ is used because the timing offset is different for each detector. Any process that produces a timing delay that does not change from pulse to pulse contributes to $C_{i}$. Examples of this are the time required for photons to travel from the bremsstrahlung radiator to the target, the propagation of signals through the wires connecting the PMTs, and delays in the electronics for processing. 

The time required for scintillation light to travel through the detector, from point of scintillation to a PMT mounted at the detector's top or bottom, can vary from 1 ns for particles that hit very close to a given PMT, to about 8 ns for particles that hit across the detector from a given PMT. The sum of the times taken for scintillation light to travel to the top and bottom PMTs is just the time taken for the light to travel the full length of the detector, which is nearly constant. The rate at which light propagates along the length of a detector is dependant on speed of light in the material and the light's flight path. The flight paths of detected scintillation light tend to be parallel to the long axis of the detector, because these paths are the shortest possible, and only the first signal from a PMT is accepted. Therefore, by taking the average of the times in the top and bottom PMT, the time required for scintillation light to propagate through the scintillator becomes a constant offset.

The validity of a constant scintillation light propagation time was verified experimentally using a $^{60}Co$ source. With the face of the scintillator covered with 2" thick bricks of lead shielding, the $^{60}Co$ source is placed against the lead at several locations along the length of the detector. The lead brick between the $^{60}Co$ source and the scintillator has a small hole drilled through it, giving the $^{60}Co$ a direct line of sight to a point on the face of the scintillator.  
%TODO: Make a figure for the co60 calibration. 

\begin{figure}
    \centering
    \includegraphics[width = 0.9\textwidth]{Content/Methods/CO60Validation.png}
    \caption{A $^{60}Co$ source, which emits coincident back-to-back photons, is placed at several positions along the face of a lead shielded scintillator. At each position, a small hole is drilled through the lead to give the $^{60}Co$ source a line-of-sight to a well-defined point on the scintillator. Then, a high timing resolution photon detector is placed close to the $^{60}Co$ source. $^{60}Co$ decays to emit two photons simultaneously--one photon is detected by the high timing resolution detector and serves as the ``start'' time, and the other photon causes a scintillation event in the detector being calibrated .}
    \label{fig:Co60Validation}
\end{figure}

\begin{figure}
    \centering
    \includegraphics[width = 0.8\textwidth]{Content/Methods/lightpaths.png}
    \caption{Hypothetical paths of light rays as they propagate through the scintillator after a scintillation event. The light rays that first reach a PMT (solid) travel a shorter path than the others (dotted), and thus experience the least amount of attenuation. This experiment always uses the first signal from a PMT and hence the data favors cases in which scintillation light travels a short path to a PMT. In other words, the detectors favor the detection of scintillation light that is traveling parallel to the long axis of the detector. As a result, the sum of the times required for scintillation light to reach both PMTs is just the time taken for light to travel the full length of the detector, which is a constant. Thus, by taking the average time between coincident events in both PMTs, the time required for the propagation of scintillation light becomes a constant that is removed during calibration. }
    \label{fig:lightpath}
\end{figure}

The value of the constant offset for the calculation of ToF is determined by observing photons that are known to have scattered from the target. Comparing the timing spectra of a non neutron producing target made from aluminum, to the spectra produced when no target is used reveals a prominent peak caused by the photons that scatter from the target. The distance these photons must travel to reach the center of a face of any detector is 125 cm, and 130 cm to reach the top or bottom edges of a detector. It takes light 4.1 ns and 4.3 ns to travel 125 cm and 130 cm, respectively. The difference between these two ToFs is negligible compared to experimental uncertainty, so the ToF of photons that scatter from the target is assumed to be 4 ns, and with this assumption the constant timing offset of each detector can be calculated.  
   
\subsection{Particle position reconstruction}
All that is known about a particle's position projected onto the horizontal plane, is that the particle is within scintillation cell of the detector in which it was detected. The detector's dimensions in the horizontal plane are comparatively small at 5X15 cm$^2$, so it suffices to use the geometric center of the detector as this component of a particle's position. Doing this creates a positional uncertainty of $\pm$8 cm, which in terms of an opening angle, amounts to $\pm4^{\circ}$. The final results of this work use an opening angle bin width of 20$^{\circ}$, so $\pm4^{\circ}$ isn't a large component of uncertainty. The largest contributor to uncertainty in the reconstruction of particle position is the position in vertical direction, which is determined by the timing difference between signals in the top and bottom PMTs. 

As is the case with ToF determination, particle position in the vertical direction relies on the timing of coincident signals from both the PMTs of a detector. If the signals from each PMT are the result of the detection of scintillation light created by the same particle, then the timing difference between the signals obeys a linear relationship with respect to the position of the particle along the vertical axis of the detector, hereafter referred to as the z-axis where the geometrical center of the detector is at $z=0$. As discussed above, scintillation light that is detected tends to travel a relatively straight path to the PMT, with minimal reflections off the boundary of the scintillation cell. Because of this, the timing difference between signals in the top and bottom PMTs is proportional to the difference in path the lengths that the scintillation light must travel to reach each PMT. The difference in paths lengths is proportional to the z coordinate, thus the timing difference is also proportional to the z coordinate. The exact linear relationship is determined through calibration by using collimated photons from a $^{60}$Co. The calibration consists of measuring the top-bottom timing difference with the source of collimated photons fixed at five different locations along the detectors length. The result is shown in figure~\ref{fig:PMTDifference}. The same setup used here was also used for ToF calibration as discussed in section~\ref{reconstruction}, in which a two inch thick brick of lead that has a hole drilled through it, about the diameter of a pencil, in order to produce a point beam of photons incident on the detector at a precise location. Noise is completely eliminated by the requirement that there be 3-fold coincidence between signals from the top and bottom PMTs of the detector being calibrated and a signal from the trigger detector. The rate. 
\begin{figure}
    \centering
    \includegraphics[width = 0.9\textwidth]{Content/Methods/PMTDifference.png}
    \caption{Caption}
    \label{fig:PMTDifference}
\end{figure}
\subsection{Detector Cross-talk}
Shielding.
POLIMI Simulation.

\subsection{Targets}
A depleted Uranium (DU) target with dimensions of XbyYbyZ $cm^3$ was used as the primary target for the measurement of two-neutron correlations. DU received the majority of the allotted beam time because it is an even-even nuclei, and as a consequence, always has exactly one quanta of angular momentum immediately upon the absorption of a photon. This makes it a good nuclei for photofission as a means to study the fundamentals of fission. 

One consideration for the dimensions of the target is the rate at which fission neutrons produced in the target scatter before exiting the target. This has the potential to be a major problem, since a scattering event changes the neutron's direction of the travel and thus creates two-neutron opening angles that are not reflective of what the opening angle was immediately after fission. This effect cannot be eliminated completely, but the target must have small enough dimensions such that the effects of neutron scattering can be neglected. The probability that a two-neutron event is undisturbed by scattering is calculated by squaring the probability that a single neutron is not disturbed. The probability that a single neutron is not disturbed is calculated by an MCNP simulation in which neutrons with an energy spectrum typical of fission neutrons are sample uniformly within a target.

It is desirable to have a target with a symmetry that is congruent with the symmetry of the cylindrically structured neutron detector array. There were difficulties in obtaining a cylindrical target, so instead a thin rectangular was rotated slowly during data acquisition. In doing so, cylindrical symmetry is preserved in the final result, since the final result is an average taken from events which occurred over a long period of time. 
The final target design was tested for neutron scattering with an MCNP simulation. In the simulation, the target cylindrical source of photons flooded inward at the target. The probability that a photo-neutron produced in the target escaped without scattering was 97.5\%. Because two neutrons are required for the formation of an opening angle, the rate of data contamination due to scattering is $(1-.975^2)$, or 5\% of two-neutron events.  


\section{Analysis}
\label{Analysis}

\subsection{Cancelation of Detector Efficiencies, Drifts, and Geometric Phase Space}
\label{subsec:SPDPCancelation}
The efficiency and acceptance of the neutron detection system varies greatly over its opening angle range of 20$^{\circ}$ to 180$^{\circ}$, as illustrated in Fig.~\ref{fig:DetAcceptance}.
This is both due to the neutron detection system's non-spherical symmetry and to varying efficiency as a function of particle position on the detector.
In order to give a result that is sensitive to angular correlations and is not sensitive to detector efficiencies and experimental drifts in PMT voltage, accelerator current, etc. , angular correlation is determined by dividing a correlated neutron distribution by an uncorrelated neutron distribution:
\begin{equation}
\label{eq:angularCorr}
\text{angular correlation }  = \frac{nn_{\text{corr}}(\theta)}{nn_{\text{uncorr}}(\theta)},
\end{equation}
where $nn_{\text{corr}}(\theta)$ is the accidental-coincidence subtracted opening angle distribution, the determination of which is described in section~\ref{Reconstruction of Accidental Coincidence}, and where $nn_{\text{uncorr}}(\theta)$\footnote{While this notation implies that coincident events are only due to neutrons, about 3\% of total  $nn_{\text{corr}}(\theta)$ events are not due to neutrons. This was determined by comparing data from a non-neutron producing Al target to that from a $^{238}$U target (see Fig.~\ref{fig:Noise})} is the uncorrelated neutron distribution, which is produced by the pairing of neutron events that occurred during different pulses.
\begin{figure}[h]
\includegraphics[width=0.9\textwidth]{Content/Methods/DetAcceptance.png}
\caption{Raw n-n opening angle measurement from the photofission of $^{238}$U. 
This distribution is highly influenced by the detection system's geometry and efficiency.
}
\label{fig:DetAcceptance}
\end{figure}

The construction of $nn_{\text{uncorr}}(\theta)$ is achieved by examining pulses in pairs of two under the requirement that both pulses occurred within 0.2 seconds of each other, which makes it very likely that they occurred under the same experimental conditions.
For each pulse-pair that has two neutron events in both pulses, all possible pairs of uncorrelated neutrons are formed (a total of 4 in this case), and the opening angle of each uncorrelated neutron-neutron (n-n) pair is calculated.
The reason for considering only pulse-pairs with two events in each pulse is to increase the percentage of pulse-pairs comprising neutrons from fission as opposed to neutrons from $(\gamma,n)$.
This matters because the detection of multiple neutrons from $(\gamma,n)$ are completely removed from $nn_{\text{corr}}(\theta)$, the numerator in Eq.~\ref{eq:angularCorr}, by the subtraction of accidental coincidences, but in order to effectively minimize the dependence of the result on detector geometry/efficiency, the numerator and denominator of Eq.~\ref{eq:angularCorr} must comprise neutron pairs with a similar energy distribution.

Because $nn_{\text{corr}}(\theta)$ is comprised of correlated n-n pairs, it is important to consider how the energies of both neutrons in n-n pairs vary together, or, in other words, the joint neutron energy distribution.
For this reason, Fig.~\ref{fig:erg_distributions} plots the distributions of two binary operations applied to the energies of n-n pairs: the mean,~$0.5(E_{1} + E_{2})$, and absolute difference, ~$|E_1 - E_2|$.
The following three sets of n-n pairs are plotted: correlated n-n pairs (~$nn_{\text{corr}}(\theta)$~), different pulse n-n pairs using only pulses that had two neutron events (~$nn_{\text{uncorr}}(\theta)$~), and different pulse n-n pairs with no restriction on the number of events in a given pulse.
From Fig.~\ref{fig:erg_distributions}, it can be concluded that the joint neutron energy distributions of n-n pairs from $nn_{\text{uncorr}}(\theta)$ and $nn_{\text{corr}}(\theta)$ are very similar.

\begin{figure}[]
\centering
    \includegraphics[width=0.99\textwidth]{Content/Methods/erg_dist(thesis).png}
    \caption{
    The energy distribution of n-n pairs from $nn_{\text{corr}}(\theta)$ must be similar to those from $nn_{\text{uncorr}}$ in order for detector efficiency to cancel in Eq.~\ref{eq:angularCorr}.
        For this reason, when determining the uncorrelated neutron distribution by pairing neutron events from different pulses, only pulses which had two neutron events are used.
    This creates a $nn_{\text{uncorr}}$ distribution that uses a higher portion of neutrons from fission relative to neutrons from ($\gamma, 1n$), leading to an energy distribution that's more similar to that of the n-n pairs comprising $nn_{\text{corr}}(\theta)$.
    }
    \label{fig:erg_distributions}
\end{figure}

Figure~\ref{fig:SPDPNormalization}(a) shows the measured $nn_{\text{corr}}(\theta)$ yield distribution of neutrons from the photofission of $^{238}$U.
The structure seen here is reflective of the underlying n-n angular correlations as well as the geometric acceptance and efficiencies of the neutron detectors.
Figure~\ref{fig:SPDPNormalization}(b) reveals how a clear picture of n-n correlations emerges when taking the ratio between $nn_{\text{corr}}(\theta)$ and $nn_{\text{uncorr}}(\theta)$.
\begin{figure}[]
\centering
    \includegraphics[width=0.7\textwidth]{Content/Methods/SPDPNormalization.png}
    \caption{(a) n-n opening angle distribution from the photofission of $^{238}$U before normalization, and, (b) after normalizing to the distribution of uncorrelated n-n events from different pulses.
    All measured neutrons have an energy greater than 0.4 MeV.}
    \label{fig:SPDPNormalization}
\end{figure}

The n-n opening angle distribution of accidental neutron coincidences from the photo-disintegration of D$_{2}$O is expected to be flat.
A cylindrical bottle with a height of 4.5~cm and a radius of 1~cm was filled with heavy water 
(D$_{2}$O) and subject to the bremsstrahlung photon beam, producing neutron coincidences at a rate of 1.4$\times10^{-4}$ per pulse.
The photo-disintegration of the deuteron, which produces only uncorrelated neutrons, is the only  neutron producing reaction involved because the bremsstrahlung endpoint was below the $(\gamma, 1n)$ threshold of $^{16}$O.
The ratio between $nn_{\text{corr}}(\theta)$ and $nn_{\text{uncorr}}(\theta)$ produces a flat curve, as expected (see Fig.~\ref{fig:D2Otheta_nn}).
\begin{figure}[h]
\includegraphics[width=0.9\textwidth]{Content/Methods/D2Otheta_nn.png}
\caption{The opening angle distribution of neutrons from the photo-disintegration of D$_{2}$O is uniform.
This is expected because the photo-disintegration of D$_{2}$O emits only a single neutron, and thus this distribution arises solely from neutron accidentals.}
\label{fig:D2Otheta_nn}
\end{figure}

\FloatBarrier
\subsection{Subtraction of Accidental Coincidences}
\label{Reconstruction of Accidental Coincidence}
The detection of two uncorrelated events in coincidence, whether caused by neutrons, photons, or noise, is referred to as an \emph{accidental}.
A small number of accidental photon events will exist in the neutron time of flight range because of the smearing of the photon peak.
These events are accidentals, because they are extremely likely to be due to photons from the beam, and not photons from fission.
There are also accidentals due to noise, which can be estimated with a non-neutron producing target made from aluminum (see Fig.~\ref{fig:Noise}).
The accelerator's current was adjusted so that there are, on average, less than 1.0 fissions per accelerator pulse, but nevertheless statistical fluctuations in the number of fissions per pulse result in accidental coincident neutrons that originated from different, and therefore, uncorrelated fissions.
There are also uncorrelated neutrons produced when multiple $(\gamma, n)$ reactions occur in a single pulse.
The $^{238}$U cross-section of $(\gamma, n)$, integrated over the relevant bremsstrahlung energy distribution, is about a factor of 5.5 times greater than it is for photofission (see Fig.~\ref{fig:CrossSection}).
As a result, it is unavoidable that there be a significant number of neutron coincidences caused by multiple $(\gamma, n)$ reactions relative to those caused by correlated fission neutrons.
If not subtracted from the result, the opening angle distribution of uncorrelated neutrons will wash out the signal from correlated neutrons. 
\begin{figure}[]
\centering
    \includegraphics[width=0.8\textwidth]{Content/Methods/Noise.png}
    \caption{An Al target was designed to scatter the same number of photons as the DU target, thus serving as an equivalent non-neutron producing target well-suited to estimate noise.
    The rate of coincident events for the Al target is 3\% that of the DU target. 
        }
    \label{fig:Noise}
\end{figure}
\begin{figure}[]
\centering
    \includegraphics[width=0.95\textwidth]{Content/Methods/CrossSections.png}
    \caption{(top) ENDF cross-sections of $(\gamma$,fiss) and direct $(\gamma$,n) and direct $(\gamma$,2n).
    (bottom) Cross-sections integrated over the simulated relative rate of bremsstrahlung photons that reach the target as a function of photon energy. The integrated cross-sections of $(\gamma, n)$ is 5.5 times greater than for $(\gamma, \text{fiss})$. }
    \label{fig:CrossSection}
\end{figure}
\begin{figure}[]
\centering
    \includegraphics[width=0.9\textwidth]{Content/Methods/NoiseSubtraction.png}
    \caption{The different pulse yield captures the effects of noise due to accidentals.
    The y-axis represents the number of coincidences in which both events had a ToF within a given 20~ns wide bin.
    Because events with a ToF above 120~ns are predominately due to noise, which correspond to a 0.5 MeV neutron, the same pulse yield is equal to 1/2 times the different pulse yield.
    }
    \label{noise_siubtraction}
\end{figure}

The raw measurement consists of a mix of correlated and accidental neutron coincidences, that is
\begin{equation}
\label{eq:corr_uncorr}
nn_{\text{raw}}(\theta)= nn_{\text{corr}}(\theta) + nn_{\text{acc}}(\theta) \,
\end{equation}
where $nn_{\text{raw}}(\theta)$ and $nn_{\text{acc}}(\theta)$ are the rates, per pulse, of the detection of neutron pairs with opening angle of $\theta$ for all events and for accidental coincident events, respectively.

Because accidental coincidences consist of two independent events, it does not matter whether the two events occurred during the same pulse or during two different pulses, given that the two different pulses occurred at around the same time and thus under the same experimental conditions.
Thus, $nn_{\scaleto{DP}{4pt}}(\theta)$ is proportional to $nn_{\text{acc}}(\theta)$.
In other words,  $nn_{\scaleto{DP}{4pt}}(\theta)$ and $nn_{\text{acc}}(\theta)$ have opening angle distributions with the same shape.
However, $nn_{\text{acc}}(\theta)$ is not equal to $nn_{\scaleto{DP}{4pt}}(\theta)$, because there are, on average, twice as many events in a pulse-pair than there are in a single pulse.
For this reason, as the following analysis shows,~$nn_{\text{acc}}(\theta) = \frac{1}{2}nn_{\scaleto{DP}{4pt}}(\theta)$.

The number of uncorrelated events detected per pulse is assumed to follow the poissonian distribution, which describes the occurrence of independent random events.
Let $\lambda$ represent the mean number of uncorrelated events per pulse.
To determine the value of $\lambda$, one needs to know whether a given coincident event is in fact an accidental, as $\lambda$ only quantifies the rate of accidental coincidences.
Such information is not known, but the largest possible value for $\lambda$ is the mean number of events per pulse, because this assumes all events are uncorrelated.
This places an upper bound on $\lambda$ of $3\times 10^{-6}$ for this work, which is small enough to neglect all terms on the order of $\lambda^3$ or greater.

The per-pulse accidental coincidence rate of individual pulses summed over all opening angle bins, denoted by $\sum_{\theta} nn_{\text{acc}}(\theta)$, is equal to the poissonian probability of there being exactly two events detected in a single pulse\footnote{For the sake of brevity, cases of greater than two-fold coincidence are not considered in this analysis, and it is not necessary because of the low detection rates during this work.
It can be shown, however, that accounting for any number of coincidences, from zero all the way up to $\infty$-fold coincident events in a pulse or pulse-pair, will give the same answer.}:
\begin{equation} \label{math:SP}
    \begin{split}
    \sum_{\theta} nn_{\text{acc}}(\theta) & = \frac{e^{-\lambda}\lambda^{2}}{2!} \\
        &\approx \frac{\lambda^2}{{2}} + \mathcal{O}(\lambda^3) \, .
    \end{split}
\end{equation}
For the case of different-pulse pairs, a ``coincidence'' occurs when there is an event in both pulses.
Cases in which there are more than two events can be neglected due to their rare occurrence in this work.
Therefore, the per-pulse rate for different-pulse pairs, again summed over all opening angle bins, is the square of the poissonian probability of there being one event:
\begin{equation} \label{math:DP}
    \begin{split}
   \sum_{\theta} nn_{\scaleto{DP}{4pt}}(\theta)&= \left(e^{-\lambda}\lambda\right)^{2} \\
    &\approx \lambda^2 + \mathcal{O}(\lambda^3) \, .
    \end{split}
\end{equation}
For the reasons explained above, $nn_{\scaleto{DP}{4pt}}(\theta)$ and $nn_{\text{acc}}(\theta)$ have the same shape, thus, from Eq.'s (\ref{math:DP}) and (\ref{math:SP}) it follows that 
\begin{equation}
\label{eq:uncorr_DP}
nn_{\text{acc}}(\theta) = \frac{1}{2}nn_{\scaleto{DP}{4pt}}(\theta) \,.
\end{equation}







\chapter{Discussion of Experimental Errors}
\label{Errors}
\section{Resolution of measurement}
The position of a detected particle is known to within a specified distance, which translates into a resolution in the measurement of the opening angle between a pair of particles.
A particle's reconstructed position along a detector's length has an error of $\pm$13 cm.
Due to the detector's 15 cm width, there is also a positional uncertainty of $\pm 7.5$ cm in the direction perpendicular along the detector's length.
The amount of uncertainty in a single two-neutron opening angle measurement is determined from the uncertainties in the positions of each detected neutron.
These position uncertainties are propagated through the formula for the calculation of opening angle, which is
\begin{displaymath}
    \theta_{nn} = \text{arccos}\left(\frac{\vec{v_{1}}^{\,}\cdot\vec{v_{2}}^{\,}}{|\vec{v_{1}}^{\,}||\vec{v_{2}}^{\,}|}\right)
\end{displaymath}
where $\vec{v_{1}}^{\,} = (x_1,y_1,z_1)$ and $\vec{v_{2}}^{\,} = (x_2,y_2,z_2)$ are the detected positions of the two neutrons.
The propagation of error through this formula is achieved by evaluating the following expression
\begin{eqnarray}
\label{eq:propagation}
 \Delta \theta_{nn} & = & \left( \left(\Delta x_1 \frac{\partial \theta}{\partial x_1}\right)^{2} + \left(\Delta y_1 \frac{\partial \theta}{\partial y_1}\right)^{2} + \left(\Delta z_1 \frac{\partial \theta}{\partial z_1}\right)^{2} + \right. \\
 & & \left. + \left(\Delta x_2 \frac{\partial \theta}{\partial x_2}\right)^{2} + \left(\Delta y_2\frac{\partial \theta}{\partial y_2}\right)^{2} + \left(\Delta z_2 \frac{\partial \theta}{\partial z_2}\right)^{2} \right) ^{\frac{1}{2}} \, ,  \nonumber
\end{eqnarray}
where the $\Delta$'s represent the uncertainty in the variable that directly follows each $\Delta$.
The values and uncertainties of all events in a given opening angle bin are fed through Eq. \ref{eq:propagation}, and then the results are averaged.
The result is shown in Fig.~\ref{fig:OpeningAngleRes} and can be interpreted as the opening angle resolution as a function of $\theta_{nn}$.
\begin{figure}[h]
    \centering
    \includegraphics[width = 0.85\textwidth]{Content/Errors/OpeningAngleUncertainty.png}
    \caption{Uncertainties in opening angle determined from the propagation of position uncertainties through the opening angle calculation.
     The uncertainty of a given opening angle measurement depends on which detectors are involved and the position of the particles on the detectors.
     For this reason, the uncertainty of measurements falling within each angle bin is a distribution, so the average uncertainties are plotted here.
    The y-axis can be viewed as a measure of angular resolution in the sense that it represents the smallest angular difference that can be considered statistically significant.
    }
    \label{fig:OpeningAngleRes}
\end{figure}

\section{Counting error}
The uncertainty in the number of observed events is always assumed to be equal to $\sqrt{N}$, as per Poissonian  statistics, where N is the number of observed events.
This value is then propagated through all the analysis procedure using the standard methods for the propagation of error.
The vertical error bars seen in all results are due solely to such counting error.

\FloatBarrier
\section{Detector Cross-talk}
\label{crosstalk}
\textit{Cross-talk} occurs when, after a particle is detected once, the same particle, by any means, causes a detection to be registered in a different detector.
For example, upon detection, a particle may undergo elastic scattering and then travel into a another detector where it is detected again, or it may produce secondary particles that are detected.
The two coincident detections of a cross-talk event are causally correlated, and thus they have the potential to contaminate the signal from correlated fission neutrons.
If both detections occur during the ToF range typical for fission neutrons, then the cross-talk event cannot be distinguished from the detection of two correlated neutrons.

Recent works that measured the two-neutron angular correlations in the spontaneous fission of $^{252}$Cf and $^{240}$Pu~\cite{Pozzi2016,Verbeke2018} addressed this effect by using an MCNP-PoLiMi simulation to estimate and then subtract cross-talk from their measurements.
In this work, the issue of cross-talk is approached differently by employing the use of detector shielding aimed at reducing cross-talk to a negligible rate.
By using shielding to reduce cross-talk, this measurement is less dependent on the details of the models used by MCNP-PoLiMi to simulate neutron transport and detection.
MCNP-PoLoMi simulations are used in this work only to verify that the effect of cross-talk is negligible.
The scintillators used here are much larger than those used in similar works, such as in refs~\cite{Pozzi2016,Verbeke2018}, allowing them to be placed much farther from the fission source without causing extremely low coincidence rates. 
An increase in the distance between the detectors and the fission source makes this measurement less sensitive to angular uncertainty, which is influenced by the uncertainty in the position of a detected particle due to, for example, the scattering of neutrons from detector shielding.
Because of this, larger amounts of shielding can be used without concern of introducing large errors.

The geometry of the neutron detection system makes it kinematically impossible for a neutron to scatter from a proton in one detector--which is the basis for scintillation--and then travel directly into another detector with enough kinetic energy to be detected a second time.
For this reason, upon being detected, a neutron must scatter from one or more intermediate nuclei, such as Pb or C, in order for it to reach a different detector with enough energy to be detected a second time.
This fact follows from the conservation of energy and momentum.
Figure~\ref{fig:CrossTalkExample} illustrates a cross-talk event due to a neutron scattering in a detector's shielding.
In order to be more convinced that such events occur at negligible rates, a detailed MCNP-PoliMi~\cite{MCNP_POLIMI} simulation was performed to model cross-talk.
\begin{figure}
    \centering
    \includegraphics[width = 0.95\textwidth]{Content/Errors/CrossTalkExample.png}
    \caption{A hypothetical example of a neutron cross-talk event.
An incoming neutron is detected and then scatters from some lead shielding nearby, which changes its direction of travel such that it enters a second detector where it is detected a second time.
The scattering of a neutron from an intermediate nucleus, in this example a lead nucleus, is kinematically required in order for cross-talk to occur in this experiment.}
    \label{fig:CrossTalkExample}
\end{figure}

\section{Simulation of Detector Cross-talk}
A simulation was performed to ensure that the detector shielding effectively reduced cross-talk to negligible levels.
The simulation included all scintillators and their shielding, supporting structures, and the concrete walls surrounding the experimental cell.
MCNP-PoliMi's built-in $^{252}$Cf spontaneous fission source was used, which emits neutrons with the correct correlations and multiplicities.
Detector response was modeled using a program included with the MCNP-PoliMi distribution called MPPost~\cite{MPPost}.
The model is based on the electron equivalent light output (MeVee) produced by particles as they undergo collisions with carbon and hydrogen within organic plastic scintillators.
A minimum deposited energy of 0.4 MeV ( 0.05 MeVee for neutrons) was assumed for detectable particles, which was chosen because the neutron detection system showed a sharp decline in detection rates for neutrons below 0.4 MeV.
For neutron collisions with hydrogen, the light output in MeVee, $L$, is calculated by the following empirically derived formula
\begin{displaymath}
L = 0.0364 E_n^2 +  0.125 E_n
\end{displaymath}
where $E_n$ is equal to the loss in the kinetic energy of the neutron due to the collision.
Neutron interactions with carbon are assumed to generate a small light output of
\begin{displaymath}
L = 0.02 E_n
\end{displaymath}
As seen in Fig.~\ref{fig:Cf252MCNPVsEXP}, this model of the detection process produces a ToF spectrum that is in good agreement with the measurement.
\begin{figure}
    \centering
    \includegraphics[width = 0.9\textwidth]{Content/Errors/Cf252MCNPVsEXP.png}
    \caption{Measured \emph{versus} simulated ToF spectrum of a $^{252}$Cf spontaneous fission source. 
    The simulation also used a detector response model based on ref~\cite{MPPost}}
    \label{fig:Cf252MCNPVsEXP}
\end{figure}

The simulation was initially performed with 5 cm of lead shielding placed behind the scintillators, and the number of cross-talk events accounted for 11\% of the total coincident neutron events.
The amount of cross-talk fell to 3\% if polyethylene was used instead of lead, which motivated the placement of 10~cm of polyethylene behind the detectors during construction.
Figure~\ref{fig:CrosstalkVScoincidence} shows the distribution of cross-talk events and true two-neutron coincidences as a function of reconstructed opening angle.
It is worth noting that, according to the simulation, the effect of cross-talk is not only small, but is also distributed over a wide range of angles rather than being concentrated around $\theta_{nn}=0$.
Angles greater than 125 degrees are not shown in Fig.~\ref{fig:CrosstalkVScoincidence}, because these cross-talk events can be readily identified in analysis by the large amount of time required for a neutron to travel the required distances.
\begin{figure}
    \centering
    \includegraphics[width = 0.95\textwidth]{Content/Errors/CrosstalkVScoincidence.png}
    \caption{
    MCNP-PoLiMi simulation of the number of cross-talk events \emph{versus} correlated two-neutron events as a function of reconstructed opening angle.
    Cross-talk accounted for 3\% of total events.
    In this work, cross-talk does not occur primarily at small angles, but is instead spread out over a wide range of angles.
    }
    \label{fig:CrosstalkVScoincidence}
\end{figure}

\section{Neutron Scattering within Target}
\label{subsection:Elastic_scattering}
A potential source of error in opening angle measurements is the scattering of neutrons within the fission target.
This is a cause for concern because when a neutron scatters from a heavy nucleons, such as $^{238}$U, it is very likely to be deflected at large angles, resulting in two-neutron opening angles that do not truly reflect the underlying fission kinematics.
%Furthermore, because the target used in this work has the shape of a thin strip, it is more likely that neutrons emitted along the wide, 2~cm, axis of the strip undergo a scattering event than those emitted along the thinest, 0.05~cm, axis.
%As a result, detectors that are located collinear to the widest axis of the target will see relatively fewer neutrons due to increased scattering. 
%This bias is removed by slowly rotating the target about the vertical axis during data acquisition.
%Because the subject of this measurement is fundamentally a statistical process, useful interpretations of the data are average rates taken over many events.
%Thus, by rotating the target, cylindrical symmetry is preserved in the average to produce a result equivalent to that if a cylindrical target were used.
The target's dimensions are small enough that the rate of photon absorption, and thus photo-neutron production, is virtually uniform throughout the entire target volume.
Accordingly, an MCNP-PoLiMi simulation was used to generate $^{252}$Cf spontaneous fission events uniformly throughout the target's volume.
The SF of $^{252}$Cf is used instead of the photofission of $^{238}$U because of the current lack of photofission models, however, the underlying fission kinematics are broadly speaking the same for the SF of $^{252}$Cf and the photofission of $^{238}$U, thus, the two process have similar neutron-neutron correlations.
The probability that at least one neutron out of a pair of two scatters before exiting the target was calculated from the simulation, indicating that 6\% of measured two-neutron opening angles were perturbed due to scattering.

The rate of elastic scattering is affected by the size and shape of the target.
A thin strip is the ideal target shape regarding the rate of neutron elastic scattering per unit of total target volume.
See Fig~\ref{fig:ElasticScatteringPlot} for the simulated elastic scattering rates for both thin strip and cylindrical shaped targets.
The simulation indicated that the rate of elastic scattering in cylindrical targets is about a factor of two times greater than in thin strip targets with the same volume and a width-to-thickness ratio of 40.

\begin{figure}
    \centering
    \includegraphics[width = 0.95\textwidth]{Content/Errors/ElasticScatteringPlot.png}
    \caption{
     Result of an MCNP simulation in which neutron-neutron pairs, with energies sampled from a typical watt fission spectrum, were generated uniformly throughout the volume of DU targets.
        The y-axis is the rate of opening angle contamination due to the scattering of, within the DU target in which they were produced, either one or both of a pair of neutrons.
    The lack of symmetry of a thin strip target can be removed by slowly rotating the target around the vertical axis during data acquisition, making it the optimal target geometry for the minimization of the rate of neutron scattering.
    The target used in this work had a length of 4~cm, a width of 2~cm, and a thickness of 0.05~cm.
    }
    \label{fig:ElasticScatteringPlot}
\end{figure}

Section~\ref{sec:anomaly} discusses the observation of an unexpected drop in correlation around 180$^{\circ}$ in our photofission of $^{238}$U measurement, as seen in Figs.~\ref{fig:DU(0)}-\ref{fig:DU(1)}.
This motivated a second simulation regarding elastic scattering which examined whether this decrease in the correlation around 180$^{\circ}$ opening angles reflects the underlying physics of the fission process.
In particular, note that throughout these measurements, the target was continuously rotated once per 8 seconds.
This means that for the determination of the uncorrelated opening angle distribution, the trajectories of the two neutrons were taken from two different pulses in which the target was at a different orientation for each of them.
Additionally, each of the neutrons likely originated from different regions of the target volume.
On the other hand, for the same-pulse, correlated neutron measurement, the target was in the same orientation and the two neutrons were generated at the same position in the target.
For these reasons, the rates of neutron scattering within the target are not necessarily equal for the same-pulse and different-pulse cases.
As such, we investigated whether these differences could cause this apparent decrease in the opening angle distribution.

Using the correlated $^{252}$Cf SF source built-in to MCNP-PoLiMi, the opening angle distribution of neutrons at the moment of emission can be compared to that of emitted neutrons after they have escaped the target.
Analysis of the simulation employs the same technique outlined in section~\ref{subsec:SPDPCancelation}, in which a correlated neutron distribution is divided by an uncorrelated neutron distribution.
The correlated opening angle distribution is formed by pairing neutrons emitted during the same fission, and the uncorrected distribution by the pairing of neutrons emitted during different fissions.
The location of fission events were sampled uniformly throughout the targets volume.
In order to account for the effect of a rotating target on the trajectories of neutrons from different-pulses, the coordinate system was rotated about the vertical axis accordingly for each fission event.

The rate of neutron scattering within the target of detected neutrons may be affected by the fact that the detection system does not have 4$\pi$ coverage. 
The potential for this to affect the measurement is captured in the simulated reconstructed $\theta_{nn}$ distribution by only counting neutrons which enter the physical volume at which a detector was located during the experiment.
A neutron energy cut of 0.4 MeV is applied to all neutrons in order to resemble the measurement.
Fig~\ref{fig:ElasticScatteringEffect} compares the two-neutron opening angle distribution at the moment immediately after emission (denoted by \emph{true} in figure) to that of neutrons once they have escaped the target (denoted by \emph{reconstructed} in figure).
The conclusion is that the rotating 0.05$\times$2$\times$4~cm$^3$ U-238 target does not, due to neutron scattering, result in a significant departure from the true opening angle distribution.
\begin{figure}
    \centering
    \includegraphics[width = 0.95\textwidth]{Content/Errors/EffectOfElasticScattering.png}
    \caption{MCNP-PoLiMi simulation of correlated $^{252}$Cf neutrons sampled uniformly throughout a 0.05$\times$2$\times$4~cm$^3$ U-238 target.
    The purpose of this simulation was to rule out errors caused by the scattering of fission neutrons within the target.
    A neutron cut of 0.4 MeV is applied to all neutrons as to more closely reflect the experiment.
    The slight difference between the curves is due solely to the elastic scattering of neutrons within the target, since detector physics was not simulated.
    However, in the reconstructed $\theta_{nn}$ distribution ({\tiny \ding{54}}), only neutrons which enter a physical volume at which a detector was located during the experiment are counted.
    All emitted neutrons greater than 0.4 MeV are counted in the true $\theta_{nn}$ distribution at emission ($\mathlarger{\mathlarger{\mathlarger{\circ}}}$).
    }
    \label{fig:ElasticScatteringEffect}
\end{figure}
\chapter{Results}
%A measurement of two-neutron opening angle distributions in the photofission of $^{238}$U
%using Bremsstrahlung photons produced via a low duty factor, pulsed LINAC, is complete.
%These results confirm the expected dependency of neutron coincidence rate on opening angle and mean neutron energy.
%Figure~\ref{fig:result} includes the coefficients from a least-squares best-fit of the data to a second order Legendre polynomial.
%These coefficients, $a_0$ and $a_2$, are a measure of the isotropic and dipole components of the opening angle distributions, respectively.
%\begin{figure}
%    \centering
%    \includegraphics[width =\textwidth]{Content/Results/result.png}
%    \caption{Preliminary results of two-neutron opening angle distributions for three different average neutron energy ranges (energy range at top of each plot).
%    The coefficients of a best-fit to a 2$^{nd}$ order Legendre polynomial appear at the bottom of each plot.
%    }
%    \label{fig:result}
%\end{figure}


\chapter{Concluding Remarks}
% TODO: Make this section complete. 
Two-neutron angular correlations in the photofission of $^{238}$U were measured using 10.5 MeV end-point bremsstrahlung photons produced via a low duty factor, pulsed linear electron accelerator.
The measured angular correlations reflect the underlying back-to-back nature of the fission fragments.
The method of analysis uses a single set of experimental data to produce a opening angle distribution of correlated and uncorrelated neutron pairs.
A ratio is taken between these two sets to provide a self-contained result of angular correlations that is independent of neutron detector efficiencies.
Two-neutron angular correlation measurements were also made using neutrons from the SF of $^{252}$Cf and are in good agreement with previous measurements.

Measured two-neutron opening angular correlations were in fair agreement with simulations that used FREYA version 2.0.3, which uses a neutron-induced model to approximate photofission.
These data will be useful for fine-tuning the photofission models that will be incorporated into future versions of FREYA.

An anomaly was observed in the rate of neutron emission at opening angles near 180$^{\circ}$, in which diminished rates resulted in a local maximum at about 160$^{\circ}$ instead of the expected 180$^{\circ}$ as seen in all past measurements of neutron-induced and spontaneous fission.
We offer two possible explanations for this effect:
\begin{enumerate*}[label=(\roman*), itemjoin={{, }}, itemjoin*={{, or }}]
  \item There is a decrease in neutron emission along the fission axis
  \item the neutrons may indeed be emitted isotropically in the rest frame of the fission fragment, but one fragment essentially shadows the neutrons emitted from the other fragment, either through absorption or scattering.
  \end{enumerate*}
A definitive interpretation of this decreased two-neutron correlation for large opening angles in photo-fission requires further study.

\appendix

\chapter*{Appendix}
\addcontentsline{toc}{chapter}{Appendix}
\thispagestyle{fancy}
\label{sec:rates}
Table~\ref{table:rates} shows the rates, per pulse, of the detection of photons and neutrons for each detector.
The overall rate of neutron singles and doubles was 4.89$\times 10^{-3}$ and 3.57$\times 10^{-5}$ per pulse, respectively.
\begin{center}
 \begin{table}
 \centering
\begin{tabular}{c  S[table-format=1.4e-2] S[table-format=1.4e-2]}%S[table-format=1.4]} %{||c |c |c |c| c||}  ccS[table-format=1.4e-2]
 \toprule
\textbf{Detector} & \textbf{neutron rate} & \textbf{photon rate} \\ [0.5ex] 
\midrule
30 bottom & 4.43E-04 & 1.93E-01 \\ \midrule
30 top & 2.11E-04 & 1.68E-01 \\ \midrule
54 & 5.03E-04 & 3.77E-01 \\ \midrule
78 & 4.27E-04 & 9.67E-02 \\ \midrule
102 & 3.61E-04 & 4.73E-02 \\ \midrule
126 & 7.13E-04 & 5.14E-02 \\ \midrule
150 & 5.76E-04 & 3.79E-02 \\ \midrule
210 & 7.16E-04 & 4.99E-02 \\ \midrule
234 & 4.49E-04 & 4.49E-02 \\ \midrule
258 & 5.27E-04 & 5.90E-02 \\ \midrule
282 & 4.42E-04 & 1.04E-01 \\ \midrule
306 & 3.40E-04 & 3.17E-01 \\ \midrule
330 bottom & 3.46E-04 & 2.35E-01 \\ \midrule
330 top & 3.24E-04 & 2.50E-01 \\
\bottomrule
\end{tabular}
\caption{
Per pulse rate of neutrons and photons on each detector.
Only one particle can be detected by a given detector per pulse, so the rate of photon detection affects the measured neutron rate.
}
  \label{table:rates}
\end{table}
\end{center}


\backmatter

\bibliographystyle{plain}
\bibliography{./refs.bib}


\end{document}